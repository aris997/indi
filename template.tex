\documentclass[11pt,a4paper,openright]{article}

\usepackage{graphicx}
\usepackage{geometry}
\usepackage[italian]{babel}
\usepackage{adjustbox}
\usepackage{amsmath}
\usepackage{float}

\begin{document}


\title{\textbf{II Prova Individuale di Laboratorio di Meccanica} \\ uN C0SA M1SuRaTa\\\small{Prof. S. De Cecco}}
\author{Riccardo Riva\\D2-08}
\date{22 giugno 2017}

\maketitle
 \newpage
\section{Introduzione}
Gli apparati a disposizione nel laboratorio per eseguire le dovute misurazioni sono:
\begin{itemize}
\item
\end{itemize}

\subsection{Stima degli errori degli strumenti}
diocare

\section{Analisi Dati}

\section{Conclusione}



\end{document}










                                      %____________________________________%
%%%%%%========  RACCOLTA DI FORMULE GIÀ TRASCRITTE  =========%%%%%%






%==== MINIMI QUADRATI ====%

\begin{eqnarray*}
m= \frac{\overline{xy}- \overline{x}\; \overline{y}}{\overline{x^2}-\overline{x}^2}\\
c=\frac{\overline{x^2}\overline{y}-\overline{x}\;\overline{xy}}{\overline{x^2}-\overline{x}^2}\\
\sigma(m)=\frac{1}{(\overline{x^2}-\overline{x}^2)\cdot \sum\nolimits_{i} \frac{1}{\sigma^2(y_i)} }\\
\sigma(c)=\frac{\overline{x^2}}{(\overline{x^2}-\overline{x}^2)\cdot \sum\nolimits_{i} \frac{1}{\sigma^2(y_i)} }\\
Cov[m,c]=\frac{\overline{x}}{(\overline{x^2}-\overline{x}^2)\cdot \sum\nolimits_{i} \frac{1}{\sigma^2(y_i)} }
\end{eqnarray*}



%==== SISTEMA DI EQUAZIONI ====%

\begin{eqnarray*}
\begin{cases}
15 + A = 36 \\
2A =6^2
\end{cases}
\end{eqnarray*}



%==== RAPPRESENTAZIONE DI UNA TABELLA ====%


\begin{table}[H]
\centering
\begin{tabular}{|c|c|}
  \hline
 colonna 1 & colonna 2\\
 \hline
 $valori$ & $valori$\\
   \hline
\end{tabular}
\end{table}



%==== FORMULA DI PROPAGAZIONE DELLE INCERTEZZE CON DEV PARZIALI ====%



\begin{eqnarray}
\delta (f)= \left|\frac{\delta f}{\delta x}\right| \sigma(x) \oplus \left|\frac{\delta f}{\delta y}\right| \sigma(y) \oplus \left|\frac{\delta f}{\delta z}\right| \sigma(z)
\end{eqnarray}



%==== GRAFICI IN PDF ====%


\begin{figure}
\label{fig:cosa}
\includegraphics[scale=0.5]{nome.pdf}
\caption{Questo testo deve avere un senso altrimenti cancella la riga}
\end{figure}

	%==== Richiamare un label ====%
	
		\ref{fig:cosa} %[per funzionare non deve avere l'asterisco la figura]
	
	
	
%==== GEOMETRIA DELLA PAGINA ====%

\newpage
\newgeometry{top=0.5cm, bottom=0.5cm, right=0.5cm, left=0.5cm}


\restoregeometry
		











